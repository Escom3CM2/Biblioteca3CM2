%========================================================
%Proceso
%========================================================

%========================================================
% Descripción general del proceso
%-----------------------------------------------
\begin{Proceso}{P4.4}{Consultar historial de Lector} {
  
  %-------------------------------------------
  %Resumen

  Proceso que realiza el \cdtRef{Actor:Lector}{Lector} desde su perfil con el fin de consultar y visualizar los datos de su historial en el cual se visualizarán los adeudos y/o multas que tenga el \cdtRef{Actor:Lector}{Lector}.
  
Si el \cdtRef{Actor:Lector}{Lector} está registrado en el sistema podrá iniciar Sesióon y tenerun perfil propio desde el cual el \cdtRef{Actor:Lector}{Lector} podrá consultar su historial.
Cuando procede la consulta el sistema generara una tabla presentando los siguientes datos: Adeudos, multas y el estado del usuario (activo o inactivo)
El sistema termina el proceso de la consulta 


  %-------------------------------------------
  %Diagrama del proceso

  \noindent La Figura \cdtRefImg{P4.4}{Consulta historial de lector} muestra las actividades que se realizan para llevar a cabo el proceso descrito anteriormente.

  \Pfig[0.95]{./procesos/C4/Images/GU4_4-CosultaLector.png}{P4.4}{Consulta historial de Lector}

} {P4.4:Consulta historial de lector}

  %-------------------------------------------
  %Elementos del proceso

  \UCitem{Actores} { %Actores
    \cdtRef{Actor:Lector}{Lector} y \cdtRef{Actor:PGU}{PGU}
  }

  \UCitem{Objetivo} { %Objetivo
    Consultar el historial de Adeudos y Multas de igual manera el estado de un usuario desde su perfil mostrando los datos en en una tabla.
  }

  \UCitem{Insumos de entrada} { %Insumos de entrada
  	\begin{UClist}
  		\UCli Clic en el Botón consultar Historial
    \end {UClist}
  }
  
  \UCitem{Proveedores} { %Proveedores
    \cdtRef{Actor:Lector}{Lector}
  }

  \UCitem{Productos de salida} { %Productos de salida
    \begin{UClist}
      \UCli Tabla generada mostrando los adeudos y multas del \cdtRef{Actor:Lector}{Lector}
      \UCli Notificación \cdtIdRef{MSJ4.6}{El lector aun no tiene historial}.
    \end{UClist}
  }

  \UCitem{Cliente} { %Cliente
    \cdtRef{Actor:Lector}{Lector}
  }

  \UCitem{Mecanismo de medición} { %Mecanismo de medición
    \begin{UClist}
      \UCli Respuesta inmediata 
    \end{UClist}
  }
  \UCitem{Interrelación con otros procesos} { %Interrelación con otros procesos
    \cdtIdRef{P4.1}{Dar de baja usuario}
  }


\end{Proceso}

%========================================================
%Descripción de tareas
%-----------------------------------------------
\begin{PDescripcion}

  %Actor: Aspirante
  \Ppaso Lector 

    \begin{enumerate}

      %Tarea a
      \Ppaso[\itarea] \cdtLabelTask{T1-P0.1:Lector}{Clic en consultar historial.} Una vez iniciada sesión el \cdtRef{Actor:Lector}{Lector} dará clic en el botón de consultar historial desde su perfil.
      
            \Ppaso[\itarea] \cdtLabelTask{T2-P4.5:Lector}{No tiene historial.} Se el mostrará la Notificación \cdtIdRef{MSJ4.6}{El lector aun no tiene historial.} 
            
            \Ppaso[\itarea] \cdtLabelTask{T3-P4.5:Lector}{Historial del Lector.} Se le mostrará la tabla del historial.              
            
    \end{enumerate}

  %Actor: SAEV2.0
  \Ppaso PGU
%\cdtRefTask{T2-P0.1:SAEV2.0}{Notifica solicitud fuera de periodo de registro.}
    \begin{enumerate}

      %Tarea a
      \Ppaso[\itarea] \cdtLabelTask{T1-P4.4:PGU}{Verificar Existencia} Para que el sistema pueda consultar y visualizar la información de el \cdtRef{Actor:Lector}{Lector} este debe verificar si el     \cdtRef{Actor:Lector}{Lector} tiene un historial de adeudos o multas de ser asi se pasa a la tarea \cdtRefTask{T2-P4.4:PGU}{Genera tabla con la información del Lector.}, de lo contrario pasa a la tarea \cdtRefTask{T3-P4.4:PGU}{Genera mensaje de inexistencia de historial.}

      %Tarea b
      \Ppaso[\itarea] \cdtLabelTask{T2-P4.4:PGU}{Genera tabla con la información del usuario.} El sistema genera una tabla con la información de los adeudos y multas asociadas al \cdtRef{Actor:Lector}{Lector} para que sea mostrada pasando a la tarea \cdtRefTask{T3-P4.5:Lector}{Información del Lector.}

      %Tarea c
      \Ppaso[\itarea] \cdtLabelTask{T3-P4.4:PGU}{Genera mensaje de inexistencia de historial.} El sistema genera el mensaje de Notificación \cdtIdRef{MSJ4.6}{El lector aun no tiene historial} para que sea mostrado al \cdtRef{Actor:Lector}{Lector} pasando a la tarea \cdtRefTask{T2-P4.5:Lector}{No tiene historial.}

      
    \end{enumerate}

\end{PDescripcion}
