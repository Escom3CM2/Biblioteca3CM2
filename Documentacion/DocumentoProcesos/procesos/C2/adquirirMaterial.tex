%-----------------------------------------------------------------------------------------------------------------------------------------------
%========================================================
%Proceso Actualizar datos de un Alumno
%========================================================

%========================================================
% Descripción general del proceso
%-----------------------------------------------
\begin{Proceso}{P2.1}{Adquisición de Material} {
  
  %-------------------------------------------
  %Resumen

  Proceso que realiza el encargado de \cdtRef{Actor:Procesos_T}{Procesos Técnicos} con la finalidad de obtener nuevos ejemplares de libros, equipo audiovisual o equipo de cómputo, que se requieren de acuerdo a las necesidades  de la biblioteca y sujeto al presupuesto establecido por Recursos Materiales del IPN.\vspace{5mm}
  
\textbf{Para adquisicón de libros y/o Material Audiovisual:} El encargado de \cdtRef{Actor:Procesos_T}{Procesos Técnicos} envía al \cdtRef{Actor:Jefe_A}{Jefe de Academia} de cada una de las asignaturas el presupuesto aprobado para cada una de las mismas, éste último se encarga de comunicar a el/los \cdtRef{Actor:Maestro}{Maestro(s)} tal situación para que lleven a cabo propuestas de libros y/o material audiovisual que posteriormente serán evaluadas de acuerdo a su nivel de importancia, una vez completadas todas las propuestas, se hará una selección de los mejores libros y/o material auidiovisual por parte del \cdtRef{Actor:Jefe_A}{Jefe de Academia}, la selección es enviada al encargado de \cdtRef{Actor:Procesos_T}{Procesos Técnicos} para que éste último organice la información y la agrupe, después de que ha sido agrupada, realiza el pedido a \cdtRef{Actor:Recursos_M}{Recursos Materiales} de la biblioteca. Al momento de que recursos materiales recibe la lista de libros y/o material audiovisual a comprar, evalua las posibilidades y cotiza con los provedores para cerciorarse de elegir la mejor opción; Envia al \cdtRef{Actor:Proveedor}{Proveedor} seleccionado la lista de los libros y/o material audiovisual a comprar, el proveedor recibe los productos y genera una factura por el monto a cubrir, recursos materiales efectua el pago y se genera un comprobante,se acuerda una fecha y hora para la entrega de los productos, recursos materiales verifica el pedido una vez que éste ha sido recibido y se envía a procesos técnicos, allí se lleva a cabo el proceso de registro y preparación de los libros y/o material audiovisual para su disponibilidad en biblioteca.
\vspace{5mm}

\textbf{Para adquisicón de equipo de cómputo y TT:} En ésta situación el encargado de \cdtRef{Actor:Procesos_T}{Procesos Técnicos} únicamente recibe el equipo de cómputo o TT a registrar, lo da de alta en el sistema y lo etiqueta para después acomodarlo en el lugar dónde sea necesario.

  
  %-------------------------------------------
  %Diagrama del proceso
  \newpage
  \noindent La Figura \cdtRefImg{P2.1}{Adquirir Libros y/o material audiovisual} muestra las actividades que se realizan para llevar a cabo el proceso descrito anteriormente.

  \Pfig[0.85]{./procesos/C2/Images/adquirirLibros.png}{P2.1}{Adquirir Libros y/o material audiovisual}

} {P2.1:Adquirir Libros y/o material audiovisual}

  %-------------------------------------------
  %Elementos del proceso
  \UCitem{Actores} { %Actores
	\cdtRef{Actor:Procesos_T}{Procesos Técnicos},
	\cdtRef{Actor:Jefe_A}{Jefe de Academia} ,
	\cdtRef{Actor:Maestro}{Maestro},
	\cdtRef{Actor:Recursos_M}{Recursos Materiales},
	\cdtRef{Actor:Proveedor}{Proveedor}
  }

  \UCitem{Objetivo} { %Objetivo
    Obtener nuevos ejemplares de libros, material audiovisual, TT y/o equipo de cómputo y darlos de alta en el sistema.
  }

  \UCitem{Insumos de entrada} { %Insumos de entrada
  	\begin{UClist}
  		\UCli Datos del Formulario \cdtIdRef{F2.1}{Presupuesto aceptado}. 
  		\UCli Datos del Formulario \cdtIdRef{F2.2}{Selección de Libros y/o material audiovisual}.
		\UCli Datos del Formulario \cdtIdRef{F2.3}{Petición de Libros y/o material audiovisual}.
		\UCli Datos del Formulario \cdtIdRef{F2.3}{Lista de Material a surtir}. 
     	
    \end {UClist}
  }
  
  \UCitem{Proveedores} { %Proveedores
    \cdtRef{Actor:Proveedor}{Proveedor}
  }

  \UCitem{Productos de salida} { %Productos de salida
    \begin{UClist}
		\UCli	Actualización de material disponible.
    \end{UClist}
  }

  \UCitem{Cliente} { %Cliente
    \cdtRef{Actor:Procesos_T}{Procesos Técnicos} y
	\cdtRef{Actor:Recursos_M}{Recursos Materiales},
  }

  \UCitem{Mecanismo de medición} { %Mecanismo de medición
    \begin{UClist}
      \UCli Respuesta inmediata
    \end{UClist}
  }
  \UCitem{Interrelación con otros procesos} { %Interrelación con otros procesos
  }


\end{Proceso}
%========================================================
%Descripción de tareas
%-----------------------------------------------
\begin{PDescripcion}

  \Ppaso Procesos Técnicos
	\begin{enumerate}
		\Ppaso[\itarea] \cdtLabelTask{T1-P2.1:Procesos Técnicos}{Recepción de Presupuesto} El \cdtRef{Actor:Bibliotecario}{Bibliotecario} recibe el comunicado de Recursos Materiales sobre el presupuesto para comprar libros e informa a los jefes de académia.
	\end{enumerate}

	\begin{enumerate}
		\Ppaso[\itarea] \cdtLabelTask{T2-P2.1:Bibliotecario}{Recepción de petición de Libros} El \cdtRef{Actor:Procesos_T}{Procesos Técnicos} recibe la lista con la petición de libros a pedir y junta en una sola lista.
	\end{enumerate}

	\begin{enumerate}
		\Ppaso[\itarea] \cdtLabelTask{T3-P2.1:Bibliotecario}{Recepción de Libros} El \cdtRef{Actor:Bibliotecario}{Bibliotecario} recibe los ejemplares comprados por Recursos Materiales y verifica la entrega para proceder a la prepraración de los libros.
	\end{enumerate}	
%%		\cdtRefTask{T2-P4.3:Bibliotecario}{Autentificar Lector.}	
\end{PDescripcion}


%------------------------------------------------------------------------------------------------------------------------------------------------


