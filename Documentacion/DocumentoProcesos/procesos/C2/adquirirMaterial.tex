%========================================================
%Proceso
%========================================================

%========================================================
% Descripción general del proceso
%-----------------------------------------------
\begin{Proceso}{P2.1}{Adquisición de Material} {
  
  %-------------------------------------------
  %Resumen
Proceso que realiza el encargado de \cdtRef{Actor:Procesos_T}{Procesos Técnicos} con la finalidad de obtener nuevos ejemplares de libros, equipo audiovisual o equipo de cómputo, que se requieren de acuerdo a las necesidades  de la biblioteca y sujeto al presupuesto establecido por Recursos Materiales del IPN.\vspace{5mm}
  
\textbf{Para adquisicón de libros y/o Material Audiovisual:} El encargado de \cdtRef{Actor:Procesos_T}{Procesos Técnicos} envía al \cdtRef{Actor:Jefe_A}{Jefe de Academia} de cada una de las asignaturas el presupuesto aprobado para cada una de las mismas, éste último se encarga de comunicar a el/los \cdtRef{Actor:Maestro}{Maestro(s)} tal situación para que lleven a cabo propuestas de libros y/o material audiovisual que posteriormente serán evaluadas de acuerdo a su nivel de importancia, una vez completadas todas las propuestas, se hará una selección de los mejores libros y/o material auidiovisual por parte del \cdtRef{Actor:Jefe_A}{Jefe de Academia}, la selección es enviada al encargado de \cdtRef{Actor:Procesos_T}{Procesos Técnicos} para que éste último organice la información y la agrupe, después de que ha sido agrupada, realiza el pedido a \cdtRef{Actor:Recursos_M}{Recursos Materiales} de la biblioteca. Al momento de que recursos materiales recibe la lista de libros y/o material audiovisual a comprar, evalua las posibilidades y cotiza con los provedores para cerciorarse de elegir la mejor opción; Envia al \cdtRef{Actor:Proveedor}{Proveedor} seleccionado la lista de los libros y/o material audiovisual a comprar, el proveedor recibe los productos y genera una factura por el monto a cubrir, recursos materiales efectua el pago y se genera un comprobante,se acuerda una fecha y hora para la entrega de los productos, recursos materiales verifica el pedido una vez que éste ha sido recibido y se envía a procesos técnicos, allí se lleva a cabo el proceso de registro y preparación de los libros y/o material audiovisual para su disponibilidad en biblioteca.
\vspace{5mm}

\textbf{Para adquisicón de equipo de cómputo y TT:} En ésta situación el encargado de \cdtRef{Actor:Procesos_T}{Procesos Técnicos} únicamente recibe el equipo de cómputo o TT a registrar, lo da de alta en el sistema y lo etiqueta para después acomodarlo en el lugar dónde sea necesario.

  %-------------------------------------------
  %Diagrama del proceso
\newpage
  \noindent La Figura \cdtRefImg{P2.1}{Adquirir Libros y/o material audiovisual} muestra las actividades que se realizan para llevar a cabo el proceso descrito anteriormente.

  \Pfig[0.85]{./procesos/C2/Images/adquirirLibros.png}{P2.1}{Adquirir Libros y/o material audiovisual}

} {P2.1:Adquirir Libros y/o material audiovisual}


  %-------------------------------------------
  %Elementos del proceso

   \UCitem{Actores} { %Actores
	\cdtRef{Actor:Procesos_T}{Procesos Técnicos},
	\cdtRef{Actor:Jefe_A}{Jefe de Academia} ,
	\cdtRef{Actor:Maestro}{Maestro},
	\cdtRef{Actor:Recursos_M}{Recursos Materiales},
	\cdtRef{Actor:Proveedor}{Proveedor}
  }

  \UCitem{Objetivo} { %Objetivo
    Obtener nuevos ejemplares de libros, material audiovisual, TT y/o equipo de cómputo y darlos de alta en el sistema.
  }

  \UCitem{Insumos de entrada} { %Insumos de entrada
  	\begin{UClist}
  		\UCli Datos del Formulario \cdtIdRef{F2.1}{Presupuesto aceptado}. 
  		\UCli Datos del Formulario \cdtIdRef{F2.2}{Selección de Libros y/o material audiovisual}.
		\UCli Datos del Formulario \cdtIdRef{F2.3}{Petición de Libros y/o material audiovisual}.
		\UCli Datos del Formulario \cdtIdRef{F2.4}{Lista de Material a surtir}. 
     	
    \end {UClist}
  }
  
  \UCitem{Proveedores} { %Proveedores
    \cdtRef{Actor:Proveedor}{Proveedor}
  }

  \UCitem{Productos de salida} { %Productos de salida
    \begin{UClist}
		\UCli	Actualización de material disponible.
    \end{UClist}
  }

  \UCitem{Cliente} { %Cliente
    \cdtRef{Actor:Procesos_T}{Procesos Técnicos} y
	\cdtRef{Actor:Recursos_M}{Recursos Materiales},
  }

  \UCitem{Mecanismo de medición} { %Mecanismo de medición
    \begin{UClist}
      \UCli Respuesta inmediata
    \end{UClist}
  }
  \UCitem{Interrelación con otros procesos} { %Interrelación con otros procesos
  }


\end{Proceso}
%========================================================
%Descripción de tareas
%-----------------------------------------------
\begin{PDescripcion}

  %Actor: Aspirante
  \Ppaso Procesos Técnicos

    \begin{enumerate}

      %Tarea a
      \Ppaso[\itarea] \cdtLabelTask{T1-P2.1:Procesos_T}{Recepción de Presupuesto} El encargado de  \cdtRef{Actor:Procesos_T}{Procesos Técnicos} recibe el comunicado de \cdtRef{Actor:Recursos_M}{Recursos Materiales} sobre el presupuesto para comprar libros y/o material audiovisual e informa a los \cdtRef{Actor:Jefe_A}{Jefes de Academia} el presupuesto asignado a cada una de éstas. Para que pueda ser llevado a cabo necesita que suceda el siguiente evento:


	%Eventos
	\begin{itemize}
	  %Evento 1
	  \item Recibe un comunicado de aprobación del presupuesto asignado para cada una de las academias.	
	\end{itemize}

      %Tarea b
      \Ppaso[\itarea] \cdtLabelTask{T2-P2.1:Procesos_T}{Recolectar información} Recolecta y agrupa la información enviada por los \cdtRef{Actor:Jefe_A}{Jefes de Academia}para posteriormente ser enviada a \cdtRef{Actor:Recursos_M}{Recursos Materiales} de la escuela para cotizaciones de compra. Para que pueda ser llevada a cabo la tarea necesita que suceda el siguiente evento:

	%Eventos
	\begin{itemize}
	  %Evento 1
	  \item Recibe el oficio \cdtIdRef{F2.2}{Selección de Libros y/o material audiovisual}. por parte de los \cdtRef{Actor:Jefe_A}{Jefes de Academia}.
	  
	\end{itemize}
	
	Una vez recibida la información se verifica la siguiente condición:
	
	\begin{itemize}
	  %Evento 1
	  \item \cdtIdRef{C2.1}{Cantidad total menor}, de lo contrario se regresará a la tarea 
\cdtRefTask{T5-P2.1:Procesos_T}{Notifica Error en selección:}
	\end{itemize}
	
	%%Tarea c
	\Ppaso[\itarea] \cdtLabelTask{T3-P2.1:Procesos_T}{Verificar entrega:}Verifica que los libros recibidos por parte de \cdtRef{Actor:Recursos_M}{Recursos Materiales} sean los mismos que se pidieron y los del recibo de compra. El evento que espera esta tarea es el siguiente.
	\begin{itemize}
	  %Evento 1
	  \item Recibe un comprobante de compra de libros y/o material audiovisual junto con los elementos comprados
	 \end{itemize}
	 
	 %%Tarea d
	\Ppaso[\itarea] \cdtLabelTask{T4-P2.1:Procesos_T}{Preparación de Material:}Prepara el material que recién se acaba de dar de alta para su uso dentro de la biblioteca, Desempaca los libros y/o material audiovisual y los da de alta en el sistema para posteriormente etiquetarlos..
	
	\begin{itemize}
	  %Evento 1
	  \item Desempaca los libros y/o material audiovisual y los da de alta en el sistema para posteriormente etiquetarlos.
	  %Evento 2
	  \item Pone sello a todos los libros
	  %Evento 3	  
	  \item Coloca etiqueta antirobo para cada producto  y forra en caso de ser necesario "sólo libros".
	  %Evento 4
	  \item Acomoda el material recien dado de alta en el lugar mas conveniente dentro de la biblioteca.
	 \end{itemize}
	 
	 \Ppaso[\itarea] \cdtLabelTask{T5-P2.1:Procesos_T}{Notifica Error en selección:}Notifica personalmente al jefe de academia el error en su presupuesto para que pueda reanalizarlo.
	

    \end{enumerate}

  %Actor: Jefe de Academia
  \Ppaso Jefe de Academia

    \begin{enumerate}

      %Tarea a
      \Ppaso[\itarea] \cdtLabelTask{T6-P2.1:Jefe_A}{Seleccionar Propuestas:} Recibe y analiza el presupuesto aceptado para su acedemía, despues analiza las propuestas de los \cdtRef{Actor:Maestro}{Maestros}  implicados en la misma y envía los libros seleccionados a conveniencia e importancia de la academia.


    \end{enumerate}
    
     %Actor: Maestro
  \Ppaso Maestro

    \begin{enumerate}

      %Tarea a
      \Ppaso[\itarea] \cdtLabelTask{T7-P2.1:Maestro}{Enviar Propuestas:} Analiza en base a sus conocimientos y experiencia que libros son requeridos para cada una de las asignaturas impartidas por la academia. Para ejecutarse necesita el siguiente evento
      \begin{itemize}
      \item Recibir una notificación por parte del jefe de academía de realizar propuestas para libros de la asignatura dónde se especializa cada maestro.
      \end{itemize}


    \end{enumerate}
    
    \Ppaso Recursos Materiales

    \begin{enumerate}

      %Tarea a
      \Ppaso[\itarea] \cdtLabelTask{T8-P2.1:Maestro}{Comprar selección:} Recibe el oficio \cdtIdRef{F2.3}{Petición de Libros y/o material audiovisual} para despues analizarlo y contactar al proveedor de preferencia, el evento necesario para que se realice esta tarea es:
      \begin{itemize}
      \item Recibir el oficio \cdtIdRef{F2.3}{Petición de Libros y/o material audiovisual}.
      \end{itemize}
 
 Despues contacta al proveedor para su compra, antes de realizar la compra se verifica la condición
 
 \begin{itemize}
	  %Evento 1
	  \item \cdtIdRef{C2.2}{Presupuesto suficiente}, de lo contrario se pasará a la tarea 
\cdtRefTask{T9-P2.1:Recursos_M}{Cambiar de Proveedor:}
	\end{itemize}
	
\Ppaso[\itarea] \cdtLabelTask{T9-P2.1:Recursos_M}{Cambiar de Proveedor:} Busca un nuevo proveedor para realizar el trato.
 

    \end{enumerate}
    
    
    
    \Ppaso Proveedor
    
    \begin{enumerate}

      %Tarea a
      \Ppaso[\itarea] \cdtLabelTask{T10-P2.1:Proveedor}{Surtir Pedido:} Una vez recibida la lista de materiales que se desean surtir se envia la cotización de precio total. Se necesita el siguiente evento.
      
      \begin{itemize}
      \item Recibir el oficio cdtIdRef{F2.3}{Lista de Material a surtir}.
      \end{itemize}
 
Después se necesita recibir el pago por el total del material solicitado, para posteriormente enviar los libros.

 

    \end{enumerate}

\end{PDescripcion}
