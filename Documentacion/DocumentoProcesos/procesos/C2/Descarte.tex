%-----------------------------------------------------------------------------------------------------------------------------------------------
%========================================================
%Proceso Actualizar datos de un Alumno
%========================================================

%========================================================
% Descripción general del proceso
%-----------------------------------------------
\begin{Proceso}{P2.4}{Descartar libros } {
  
  %-------------------------------------------
  %Resumen

  Proceso que realiza el \cdtRef{Actor:Bibliotecario}{Bibliotecario} con la finalidad de llevar a cabo el descarte de libros por diversas razones al final de su vida util o por defectos que conlleva a dar de baja un libro del inventario.
  Se reúnen los libros que se encuentren en algún estado de los siguientes:
  -obsolescencia.
  -falta de uso.
  -contaminación y/o mutilación.
Despues se informa a la oficina de recursos materiales del ipn el problema de libros con eso recibe un formato de la oficina de recursos el cual contiene nombre título autores edición y motivo de descarte por cada isbn se necesita llenar el formato anterior.
Despues los libros se guardan en una caja y se envían junto con su formato pegado en la caja a recursos materiales del ipn con ello la biblioteca  es informada mediante un acuse de recibo que los libros han sido entregados satisfactoriamente.



  %-------------------------------------------
  %Diagrama del proceso

  \noindent La Figura \cdtRefImg{P2.4}{Descartar libros} muestra las actividades que se realizan para llevar a cabo el proceso descrito anteriormente.

  \Pfig[0.95]{./procesos/C2/Images/GU4_6Descartar.png}{P4.6}{Descartar libros}
\Pfig[0.95]{./procesos/Ejemplo/Images/PA2_1-SolicitudDeCuenta.png}{P0.1}{Solicitud de cuenta}

} {P4.6:Realizar Descarte de libros}

  %-------------------------------------------
  %Elementos del proceso

  \UCitem{Actores} { %Actores
    \cdtRef{Actor:Bibliotecario}{Bibliotecario} .
  }

  \UCitem{Objetivo} { %Objetivo
	Descartar un libro para darlo de baja del sistema en el inventario.
  }

  \UCitem{Insumos de entrada} { %Insumos de entrada
  	\begin{UClist}
  		\UCli Recuperacion de los libros para descartar dependiendo de la informacion de el estado de los libros.
     	
    \end {UClist}
  }
  
  \UCitem{Proveedores} { %Proveedores
    Sistema
  }

  \UCitem{Productos de salida} { %Productos de salida
    \begin{UClist}
    
      \UCli Acuse de recibido.
      \UCli Libros dados de baja del sistema.
  
    \end{UClist}
  }

  \UCitem{Cliente} { %Cliente
    \cdtRef{Actor:Bibliotecario}{Bibliotecario}
  }

  \UCitem{Mecanismo de medición} { %Mecanismo de medición
  
  }
  \UCitem{Interrelación con otros procesos} { %Interrelación con otros procesos
  }


\end{Proceso}

%========================================================
%Descripción de tareas
%-----------------------------------------------
\begin{PDescripcion}

  %Actor: Aspirante
  \Ppaso Encargado de biblioteca

    \begin{enumerate}

      %Tarea a
      \Ppaso[\itarea] \cdtLabelTask{T1-P4.6:Encargado}{Solicitar descarte de libros.}El encargado solicita iniciar el descarte de libros.
	

    \end{enumerate}
    
    
      %Actor: SAEV2.0
  \Ppaso Bibliotecario 

    \begin{enumerate}

      %Tarea a
      \Ppaso[\itarea] \cdtLabelTask{T1-P4.3:Bibliotecario}{.}Se revisa los estados de los libros que se pueden descartar por obsolescencia, falta de uso, contaminación y/o mutilación.

      %Tarea b
      \Ppaso[\itarea] \cdtLabelTask{T2-P4.3:Bibliotecario}{.}se reúnen los libros que se encuentren en algún estado anterior o que reúnan alguna de las características mencionadas anteriormente.
      
      %Tarea c
      \Ppaso[\itarea] \cdtLabelTask{T3-P4.3:Bibliotecario}{.}se informa a la oficina de recursos materiales del ipn el problema de libros.

      %Tarea d
      \Ppaso[\itarea] \cdtLabelTask{T4-P4.3:Bibliotecario}{.}se recibe un formato de la oficina de recursos el cual contiene nombre título autores edición y motivo de descarte.
      
      
      %Tarea e
      \Ppaso[\itarea] \cdtLabelTask{T5-P4.3:Bibliotecario}{.}los libros se guardaran en una caja y se envían junto con su formato pegado en la caja a recursos materiales del ipn.

      %Tarea f
      \Ppaso[\itarea] \cdtLabelTask{T3-P4.3:Bibliotecario}{.}el bibliotecario  es informado mediante un acuse de recibo que los libros han sido entregados satisfactoriamente.
      
          \end{enumerate}

\end{PDescripcion}


%------------------------------------------------------------------------------------------------------------------------------------------------
