%%%%%%%%%%%%%%%%%%%%%%%%%%%%%%%%%%%%%%%%%%%%%%%%%%%%%%%%%%%%%%%%
% Tipo de documento y paquetes
\documentclass[10pt]{book}
\usepackage{cdt/cdtAnalisis}
\usepackage{subfigure}
\usepackage{appendix}

%%%%%%%%%%%%%%%%%%%%%%%%%%%%%%%%%%%%%%%%%%%%%%%%%%%%%%%%%%%%%%%%
% Colores
\let\TAB\tabular
\renewcommand\tabular{\noindent\TAB}
\definecolor{gray1}{gray}{0.89}%{0.80}
\definecolor{gray2}{gray}{0.68}
\definecolor{gray3}{gray}{0.97}
\definecolor{pink1}{rgb}{1,0.87,0.75}
\definecolor{green1}{rgb}{0,0.75,0.75}
\definecolor{blue1}{rgb}{0.75,0.75,1}

%%%%%%%%%%%%%%%%%%%%%%%%%%%%%%%%%%%%%%%%%%%%%%%%%%%%%%%%%%%%%%%%
% Datos del proyecto
\sistema{Análisis del proceso de Gestión Escolar de la Escuela Libre de Derecho}

\proyecto[SAEV2.0]{Sistema de Administración Escolar V2.0}

\documento{Inicial}{Externo}{REING}{Propuesta del Nuevo Proceso de Admisión a Nivel Superior}{1.0}

\fecha{\today}

\organizacion{Escuela Libre de Derecho}

\author{Escuela Superior de Cómputo del IPN}

\elaboro[Líder de proyecto IPN-ESCOM]{M. en C. José Jaime López Rabadán.}   % Responsable del contenido (IPN)
\superviso[Área]{Persona.} % Quien recibe el documento (Contraparte)
\aprobo[Área]{Persona.} % Responsable Técnico (Contraparte)

\title{\varProyecto}
\subtitle {\varCveDocumento--\varDocumento}

%%%%%%%%%%%%%%%%%%%%%%%%%%%%%%%%%%%%%%%%%%%%%%%%%%%%%%%%%%%%%%%
% Elementos contenidos en el documento

% TODO: Al finalizar el análisis resuma aquí todos los elementos del componente: RN, CU, IU, MSG.
% \elemRefs{
% 	\elemItem{PU1}{1.0}{Proceso de Usuario 1, registro de nuevo Usuario}
% 	\elemItem{PPS1}{1.0}{Persona Física con Perfil Empresarial}
% }

%%%%%%%%%%%%%%%%%%%%%%%%%%%%%%%%%%%%%%%%%%%%%%%%%%%%%%%%%%%%%%%%
% Documentos relacionados con el documento actual
% TODO: Escriba los documentos en los que esta basado este documento.
\docRefs{
	\docItem{Reglamento General}{}{Reglameto General, Aprobado por la Asamblea Extraordinaria de la Junta General de Profesores, 19 de octubre de 2005. Última reforma, 27-V-2015}
	\docItem{Convocatoria de Ingreso}{}{Convocatoria de Selección a la Carrera de Abogado, Escuela Libre de Derecho, Ciclo Escolar 2017-2018 }
	\docItem{Introducción BPMN}{}{Stephen A. White. Introduction to BPMN. IBM Corporation}
	\docItem{Documentación BPMN}{}{Business Process Model and Notation (BPMN), v2.0. Número de documento en OMG: dtc/2009-08-14. URL de la documentación estándar: \url{http://www.omg.org/spec/BPMN/2.0/}. Agosto 2009}
	\docItem{F1.4-1}{1.0}{Formato de Datos Personales}
}

%%%%%%%%%%%%%%%%%%%%%%%%%%%%%%%%%%%%%%%%%%%%%%%%%%%%%%%%%%%%%%%%
% Inicio del Documento
\begin{document}

    %=========================================================
    % Portada
    \ThisLRCornerWallPaper{1}{cdt/theme/agua.jpg}
    \thispagestyle{empty}

    \maketitle
    
    %=========================================================
    % Hoja de revisión
    %\makeDocInfo
    %\bigskip\\
    %\makeElemRefs   --Coment--
    %\makeDocRefs
    %\makeObservaciones[3cm]
    %\vspace{2cm}
    %\makeFirmas

    %=========================================================
    % Indices del documento
    \frontmatter
    \LRCornerWallPaper{1}{cdt/theme/pleca.jpg}
    \tableofcontents
    \listoffigures
    %\listoftables
    \mainmatter

    %=========================================================
    % Para ocultar la información del documentador se descomenta: \hideControlVersion
    %\hideControlVersion
 
    %=========================================================
    % Capítulos del documento

    %---------- Introducción al contenido del documento
    %\input{introduccion} % Introducción al contenido del documento

    %---------- Proceso de Admisión

	%======================EJEMPLO===================================
	\chapter{Procesos}    
    \input{procesos/Ejemplo/proceso} 		%PA2.1 Generación de cuenta 
   

%====================CELULA1=====================================
%Introducir los correspondientes a la GESTIÓN DE EMPLEADOS



%====================CELULA2=====================================
%Introducir los correspondientes a la GESTIÓN DE INVENTARIO
	 %-----------------------------------------------------------------------------------------------------------------------------------------------
%========================================================
%Proceso Actualizar datos de un Alumno
%========================================================

%========================================================
% Descripción general del proceso
%-----------------------------------------------
\begin{Proceso}{P2.1}{Adquisición de Material} {
  
  %-------------------------------------------
  %Resumen

  Proceso que realiza el encargado de \cdtRef{Actor:Procesos_T}{Procesos Técnicos} con la finalidad de obtener nuevos ejemplares de libros, equipo audiovisual o equipo de cómputo, que se requieren de acuerdo a las necesidades  de la biblioteca y sujeto al presupuesto establecido por Recursos Materiales del IPN.\vspace{5mm}
  
\textbf{Para adquisicón de libros y/o Material Audiovisual:} El encargado de \cdtRef{Actor:Procesos_T}{Procesos Técnicos} envía al \cdtRef{Actor:Jefe_A}{Jefe de Academia} de cada una de las asignaturas el presupuesto aprobado para cada una de las mismas, éste último se encarga de comunicar a el/los \cdtRef{Actor:Maestro}{Maestro(s)} tal situación para que lleven a cabo propuestas de libros y/o material audiovisual que posteriormente serán evaluadas de acuerdo a su nivel de importancia, una vez completadas todas las propuestas, se hará una selección de los mejores libros y/o material auidiovisual por parte del \cdtRef{Actor:Jefe_A}{Jefe de Academia}, la selección es enviada al encargado de \cdtRef{Actor:Procesos_T}{Procesos Técnicos} para que éste último organice la información y la agrupe, después de que ha sido agrupada, realiza el pedido a \cdtRef{Actor:Recursos_M}{Recursos Materiales} de la biblioteca. Al momento de que recursos materiales recibe la lista de libros y/o material audiovisual a comprar, evalua las posibilidades y cotiza con los provedores para cerciorarse de elegir la mejor opción; Envia al \cdtRef{Actor:Proveedor}{Proveedor} seleccionado la lista de los libros y/o material audiovisual a comprar, el proveedor recibe los productos y genera una factura por el monto a cubrir, recursos materiales efectua el pago y se genera un comprobante,se acuerda una fecha y hora para la entrega de los productos, recursos materiales verifica el pedido una vez que éste ha sido recibido y se envía a procesos técnicos, allí se lleva a cabo el proceso de registro y preparación de los libros y/o material audiovisual para su disponibilidad en biblioteca.
\vspace{5mm}

\textbf{Para adquisicón de equipo de cómputo y TT:} En ésta situación el encargado de \cdtRef{Actor:Procesos_T}{Procesos Técnicos} únicamente recibe el equipo de cómputo o TT a registrar, lo da de alta en el sistema y lo etiqueta para después acomodarlo en el lugar dónde sea necesario.

  
  %-------------------------------------------
  %Diagrama del proceso
  \newpage
  \noindent La Figura \cdtRefImg{P2.1}{Adquirir Libros y/o material audiovisual} muestra las actividades que se realizan para llevar a cabo el proceso descrito anteriormente.

  \Pfig[0.85]{./procesos/C2/Images/adquirirLibros.png}{P2.1}{Adquirir Libros y/o material audiovisual}

} {P2.1:Adquirir Libros y/o material audiovisual}

  %-------------------------------------------
  %Elementos del proceso
  \UCitem{Actores} { %Actores
	\cdtRef{Actor:Procesos_T}{Procesos Técnicos},
	\cdtRef{Actor:Jefe_A}{Jefe de Academia} ,
	\cdtRef{Actor:Maestro}{Maestro},
	\cdtRef{Actor:Recursos_M}{Recursos Materiales},
	\cdtRef{Actor:Proveedor}{Proveedor}
  }

  \UCitem{Objetivo} { %Objetivo
    Obtener nuevos ejemplares de libros, material audiovisual, TT y/o equipo de cómputo y darlos de alta en el sistema.
  }

  \UCitem{Insumos de entrada} { %Insumos de entrada
  	\begin{UClist}
  		\UCli Datos del Formulario \cdtIdRef{F2.1}{Presupuesto aceptado}. 
  		\UCli Datos del Formulario \cdtIdRef{F2.2}{Selección de Libros y/o material audiovisual}.
		\UCli Datos del Formulario \cdtIdRef{F2.3}{Petición de Libros y/o material audiovisual}.
		\UCli Datos del Formulario \cdtIdRef{F2.3}{Lista de Material a surtir}. 
     	
    \end {UClist}
  }
  
  \UCitem{Proveedores} { %Proveedores
    \cdtRef{Actor:Proveedor}{Proveedor}
  }

  \UCitem{Productos de salida} { %Productos de salida
    \begin{UClist}
		\UCli	Actualización de material disponible.
    \end{UClist}
  }

  \UCitem{Cliente} { %Cliente
    \cdtRef{Actor:Bibliotecario}{Bibliotecario}
  }

  \UCitem{Mecanismo de medición} { %Mecanismo de medición
    \begin{UClist}
      \UCli Respuesta inmediata
    \end{UClist}
  }
  \UCitem{Interrelación con otros procesos} { %Interrelación con otros procesos
  }


\end{Proceso}
%========================================================
%Descripción de tareas
%-----------------------------------------------
\begin{PDescripcion}

  \Ppaso Bibliotecario
	\begin{enumerate}
		\Ppaso[\itarea] \cdtLabelTask{T1-P2.1:Bibliotecario}{Recepción de Presupuesto} El \cdtRef{Actor:Bibliotecario}{Bibliotecario} recibe el comunicado de Recursos Materiales sobre el presupuesto para comprar libros e informa a los jefes de académia.
	\end{enumerate}

	\begin{enumerate}
		\Ppaso[\itarea] \cdtLabelTask{T2-P2.1:Bibliotecario}{Recepción de petición de Libros} El \cdtRef{Actor:Bibliotecario}{Bibliotecario} recibe la lista con la petición de libros a pedir y junta en una sola lista.
	\end{enumerate}

	\begin{enumerate}
		\Ppaso[\itarea] \cdtLabelTask{T3-P2.1:Bibliotecario}{Recepción de Libros} El \cdtRef{Actor:Bibliotecario}{Bibliotecario} recibe los ejemplares comprados por Recursos Materiales y verifica la entrega para proceder a la prepraración de los libros.
	\end{enumerate}	
%%		\cdtRefTask{T2-P4.3:Bibliotecario}{Autentificar Lector.}	
\end{PDescripcion}


%------------------------------------------------------------------------------------------------------------------------------------------------



	 %-----------------------------------------------------------------------------------------------------------------------------------------------
%========================================================
%Proceso Actualizar datos de un Alumno
%========================================================

%========================================================
% Descripción general del proceso
%-----------------------------------------------
\begin{Proceso}{P2.2}{Cotejar Existencia de Material} {
  
  %-------------------------------------------
  %Resumen

  Proceso que realiza el  encargado de \cdtRef{Actor:Procesos_T}{Procesos Técnicos} cada año con la finalidad de conocer el comparativo de los libros físicos, material audiovisual y/o Equipo de Cómputo que se tienen en biblioteca con los materiales que debería haber de acuerdoa la última entrada de ejemplares "último estado de la base de datos".
  
  El encargado de \cdtRef{Actor:Procesos_T}{Procesos Técnicos} procede a cerrar la biblioteca para limpiar los materiales que desee inventariar, estos pueden ser material audiovisual, libros o equipo de cómputo y revisar su estado físico,para después obtener su código e introducirlo en el sistema donde se crea un formato de inventario con los códigos.  Éste formato se envía al sistema interno del IPN donde se realiza el comparativo y se devuelve el oficio con las diferencias de inventario.
  
  Se identifican los materiales que están reportados como faltantes, si éstos materiales son libros o material audiovisual, se verifica si un alumno posee el material, de ser así se agrega como parte del inventario y en caso contrario se realiza una búsqueda extenuante, si después de la búsqueda no se encuentra el material, se cambia el estado del material a extraviado.
En el supuesto cado de ser equipo de cómputo se vuelve a buscar en la biblioteca, de no ser encontrado se actualiza el estado a extraviado.
  
  %-------------------------------------------
  %Diagrama del proceso
  \noindent La Figura \cdtRefImg{P2.2}{Cotejar Existencia de Material} muestra las actividades que se realizan para llevar a cabo el proceso descrito anteriormente.

  \Pfig[1.0]{./procesos/C2/Images/cotejarExistencias.png}{P2.2}{Cotejar Existencia de Material}

} {P2.2:Cotejar Existencia de Material}

  %-------------------------------------------
  %Elementos del proceso
  \UCitem{Actores} { %Actores
	\cdtRef{Actor:Procesos_T}{Procesos Técnicos}
  }

  \UCitem{Objetivo} { %Objetivo
    Comparar la existencia de material en biblioteca con el último estado de la base de datos.
  }

  \UCitem{Insumos de entrada} { %Insumos de entrada
  	\begin{UClist}
  		\UCli Identificadores de material proporcionados por el \cdtRef{Actor:Procesos_T}{Procesos Técnicos} al ser escaneados:
		\UCli \cdtIdRef{F2.5}{Escanear Libros}.
		\UCli \cdtIdRef{F2.6}{Escanear Material Audiovisual}.
		\UCli \cdtIdRef{F2.7}{Escanear Equipo de Cómputo}. 
				  		 
  		 
     	
    \end {UClist}
  }
  
  \UCitem{Proveedores} { %Proveedores
    \cdtRef{Actor:Procesos_T}{Procesos Técnicos}
  }

  \UCitem{Productos de salida} { %Productos de salida
    \begin{UClist}
		\UCli	Documentos generados de inventario.
		\UCli \cdtIdRef{D2.1}{Libros Escaneados}
		\UCli \cdtIdRef{D2.2}{Material Audiovisual Escaneado}.
		\UCli \cdtIdRef{D2.3}{Equipo de Cómputo Escaneado}
    \end{UClist}
  }

  \UCitem{Cliente} { %Cliente
    \cdtRef{Actor:Bibliotecario}{Bibliotecario}
  }

  \UCitem{Mecanismo de medición} { %Mecanismo de medición
    \begin{UClist}
      \UCli Respuesta inmediata
    \end{UClist}
  }
  \UCitem{Interrelación con otros procesos} { %Interrelación con otros procesos
  	    \begin{UClist}
  	    	\UCli Préstamo
  	    \end{UClist}
  }


\end{Proceso}
%========================================================
%Descripción de tareas
%-----------------------------------------------
\begin{PDescripcion}

  \Ppaso Bibliotecario
	\begin{enumerate}
		\Ppaso[\itarea] \cdtLabelTask{T1-P2.2:Bibliotecario}{Entradas de Identificadores} El \cdtRef{Actor:Bibliotecario}{Bibliotecario} cierra la biblioteca y comienza el proceso de inventario introduciendo los identificadores de los libros en el sistema.
		
		\Ppaso[\itarea] \cdtLabelTask{T2-P2.2:Bibliotecario}{Envío y Recepción de Comparativo} El \cdtRef{Actor:Bibliotecario}{Bibliotecario} envía al servidor interno del instituto el reporte de los libros de los que se han introducido su identificador. Recibe como respuesta la lista de libros faltantes.
		
		\Ppaso[\itarea] \cdtLabelTask{T3-P2.2:Bibliotecario}{Identificación de Libros Faltantes} El \cdtRef{Actor:Bibliotecario}{Bibliotecario} consulta a préstamos los libros faltantes. Si los posee un alumno el libro se agrega al inventario, de lo contrario se reporta como perdido.
	\end{enumerate}
%%		\cdtRefTask{T2-P4.3:Bibliotecario}{Autentificar Lector.}	
\end{PDescripcion}


%------------------------------------------------------------------------------------------------------------------------------------------------





%====================CELULA3=====================================
%Introducir los correspondientes a la GESTIÓN DE PRÉSTAMOS



%====================CELULA4=====================================
%Introducir los correspondientes a la GESTIÓN DE USUARIOS

	

%====================CELULA5=====================================
%Introducir los correspondientes a la GESTIÓN DE CREDENCIALES


    
    
    %---------- Actores
    %=============================================
% Descripción de actores

\chapter{Actores del sistema}
\label{chapter:ActoresDelSistema}

En el presente capítulo se definen los actores que participan en el sistema.


%==============Alumno=========================================
\cdtLabel{Actor:Lector}{}
\begin{Actor}{Lector}{Persona que puede ser un alumno o docente que hace uso de los servicios que la Biblioteca ofrece.}
	\item[Área:] No aplica
	\item[Responsabilidades:] \hspace{1pt}
	\begin{itemize}
		\item Proporcionar datos de identificación
		\item Consultar perfil
		\item Pagar Multas
		\item Devolver Material
	\end{itemize}
	\item[Perfil:] \hspace{1pt}
	\begin{itemize}
		\item Debe ser una persona perteneciente al instituto. 		
		\item Debe ser responsable.
		\item Debe ser cumplido.
	\end{itemize}
\end{Actor}

%==============  Bibliotecario ========================
\cdtLabel{Actor:Bibliotecario}{}
\begin{Actor}{Bibliotecario}{Persona encargada de gestionar el proceso interno de la gestión de usuarios. }
	\item[Área:] ---
	\item[Responsabilidades:] \hspace{1pt}
	\begin{itemize}
		\item Dar de Alta lector
		\item Dar de baja Lector
		\item Modificar datos del Lector
			\end{itemize}
\item[Perfil:] \hspace{1pt}
	\begin{itemize}
	\item Debe ser una persona perteneciente al instituto.
		\item Debe contar facilidad de palabras
		\item Capaz de trabajar trabajar bajo presión 
		\item Debe ser egresado o estar cursando ultimo año de nivel superior 		
		\item Debe ser responsable.
	\end{itemize}
\end{Actor}




%==============  PGU ========================
\cdtLabel{Actor:PGU}{}
\begin{Actor}{PGU}{Sistema de administración bibliotecario encargado de automatizar todo lo relacionado con la gestion de usuarios.}
	\item[Área:] ---
	\item[Responsabilidades:] \hspace{1pt}
	\begin{itemize}
		\item Verificar existencia de Lectores
		\item Verificar existencia de historiales de los lectores
		\item Generar tablas con la información asociada a los lectores
			\end{itemize}
\end{Actor}



 %Actores identificados y su descripción

    %---------- Glosario
    \input{glosario}

    %---------- Anexos
    \appendix
    \cfinput{anexos/Condiciones} % Condiciones y Reglas de Negocio
    \cfinput{anexos/Mensajes} % Mensajes
    \cfinput{anexos/Formularios} % Formularios
    \cfinput{anexos/Documentos} % Documentos

    \clossing

\end{document}
